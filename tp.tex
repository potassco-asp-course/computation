% ----------------------------------------------------------------------
\begin{frame}{Consequence operator}
  \medskip
  Let $P$ be a positive program and $X$ a set of atoms
  \begin{itemize}\normalsize
  \item<1->
    The \alert{consequence operator} \Top{P} is defined as follows:
    \[
      \T{P}{X} = \{  \head{r}\mid r\in P, \pbody{r}\subseteq X  \}
    \]
  \item<only@2> \structure{Procedural characterization}
    \smallskip
  \item<only@2> [] \textbf{let} $\T{P}{X}=\{\head{r}\mid r\in P, \pbody{r}\subseteq X\}$ \textbf{in} \
    \begin{itemize}\normalsize
    \item []$X := \emptyset$
      \smallskip
    \item []\textbf{while} $\T{P}{X}\neq X$
      \begin{itemize}\normalsize
      \item[] $X := \,\T{P}{X}$
      \end{itemize}
    \item[] \textbf{return} $X$
    \end{itemize}
  \item<only@3->
    Iterated applications of $\Top{P}$ are written as $\Topi{j}{P}$ for
    $j\geq 0$
    \begin{itemize}\normalsize
    \item
      \(
      \Ti{0}{P}{X}=X
      \)
      \smallskip
    \item
      \(
      \Ti{i}{P}{X}=\T{P}{\Ti{i-1}{P}{X}}
      \)
      for $i\geq 1$
    \end{itemize}
    \medskip
\item<only@4-> \structure{Properties} \ For any positive program $P$
  \begin{itemize}\normalsize
  \item $\Cn{P}=\bigcup_{i\geq 0}\Ti{i}{P}{\emptyset}$
  \item $X\subseteq Y$ implies $\T{P}{X}\subseteq\T{P}{Y}$
  \item \Cn{P} is the smallest fixpoint of \Top{P}
  \end{itemize}
\end{itemize}
\end{frame}
% ----------------------------------------------------------------------
\begin{frame}{An example}
  \begin{itemize}
  \item
    \(
    P
    =
    \left\{
      p \leftarrow ,           \
      q \leftarrow ,           \
      r \leftarrow p,          \
      s \leftarrow q, t,       \
      t \leftarrow r,          \
      u \leftarrow v
    \right\}
    \)
  \item<2-> We get
    \[
    \begin{array}{lclclcl}
      \Ti{0}{P}{\emptyset}&=&\emptyset    & &                           & &                    \\
      \Ti{1}{P}{\emptyset}&=&\{p,q\}      &=&\T{P}{\Ti{0}{P}{\emptyset}}&=&\T{P}{\emptyset}    \\
      \Ti{2}{P}{\emptyset}&=&\{p,q,r\}    &=&\T{P}{\Ti{1}{P}{\emptyset}}&=&\T{P}{\{p,q\}}      \\
      \Ti{3}{P}{\emptyset}&=&\{p,q,r,t\}  &=&\T{P}{\Ti{2}{P}{\emptyset}}&=&\T{P}{\{p,q,r\}}    \\
      \Ti{4}{P}{\emptyset}&=&\{p,q,r,t,s\}&=&\T{P}{\Ti{3}{P}{\emptyset}}&=&\T{P}{\{p,q,r,t\}}  \\
      \Ti{5}{P}{\emptyset}&=&\{p,q,r,t,s\}&=&\T{P}{\Ti{4}{P}{\emptyset}}&=&\T{P}{\{p,q,r,t,s\}}\\
      \Ti{6}{P}{\emptyset}&=&\{p,q,r,t,s\}&=&\T{P}{\Ti{5}{P}{\emptyset}}&=&\T{P}{\{p,q,r,t,s\}}
    \end{array}
    \]
  \item<3-> $\Cn{P}=\{p,q,r,t,s\}$ is the smallest fixpoint of \Top{P}
    because
    \begin{itemize}\normalsize
    \item $\T{P}{\{p,q,r,t,s\}}=\{p,q,r,t,s\}$ and
    \item $\T{P}{X}\neq X$ for each $X\subset  \{p,q,r,t,s\}$
    \end{itemize}
  \end{itemize}
  \smallskip
\end{frame}
% ----------------------------------------------------------------------
%
%%% Local Variables:
%%% mode: latex
%%% TeX-master: "../../main"
%%% End:
